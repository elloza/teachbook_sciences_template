%% Generated by Sphinx.
\def\sphinxdocclass{jupyterBook}
\documentclass[letterpaper,10pt,english]{jupyterBook}
\ifdefined\pdfpxdimen
   \let\sphinxpxdimen\pdfpxdimen\else\newdimen\sphinxpxdimen
\fi \sphinxpxdimen=.75bp\relax
\ifdefined\pdfimageresolution
    \pdfimageresolution= \numexpr \dimexpr1in\relax/\sphinxpxdimen\relax
\fi
%% let collapsible pdf bookmarks panel have high depth per default
\PassOptionsToPackage{bookmarksdepth=5}{hyperref}
%% turn off hyperref patch of \index as sphinx.xdy xindy module takes care of
%% suitable \hyperpage mark-up, working around hyperref-xindy incompatibility
\PassOptionsToPackage{hyperindex=false}{hyperref}
%% memoir class requires extra handling
\makeatletter\@ifclassloaded{memoir}
{\ifdefined\memhyperindexfalse\memhyperindexfalse\fi}{}\makeatother


\PassOptionsToPackage{warn}{textcomp}

\catcode`^^^^00a0\active\protected\def^^^^00a0{\leavevmode\nobreak\ }
\usepackage{cmap}
\usepackage{fontspec}
\defaultfontfeatures[\rmfamily,\sffamily,\ttfamily]{}
\usepackage{amsmath,amssymb,amstext}
\usepackage{polyglossia}
\setmainlanguage{english}



\setmainfont{FreeSerif}[
  Extension      = .otf,
  UprightFont    = *,
  ItalicFont     = *Italic,
  BoldFont       = *Bold,
  BoldItalicFont = *BoldItalic
]
\setsansfont{FreeSans}[
  Extension      = .otf,
  UprightFont    = *,
  ItalicFont     = *Oblique,
  BoldFont       = *Bold,
  BoldItalicFont = *BoldOblique,
]
\setmonofont{FreeMono}[
  Extension      = .otf,
  UprightFont    = *,
  ItalicFont     = *Oblique,
  BoldFont       = *Bold,
  BoldItalicFont = *BoldOblique,
]



\usepackage[Bjarne]{fncychap}
\usepackage[,numfigreset=1,mathnumfig]{sphinx}

\fvset{fontsize=\small}
\usepackage{geometry}


% Include hyperref last.
\usepackage{hyperref}
% Fix anchor placement for figures with captions.
\usepackage{hypcap}% it must be loaded after hyperref.
% Set up styles of URL: it should be placed after hyperref.
\urlstyle{same}

\addto\captionsenglish{\renewcommand{\contentsname}{Tutorial}}

\usepackage{sphinxmessages}



        % Start of preamble defined in sphinx-jupyterbook-latex %
         \usepackage[Latin,Greek]{ucharclasses}
        \usepackage{unicode-math}
        % fixing title of the toc
        \addto\captionsenglish{\renewcommand{\contentsname}{Contents}}
        \hypersetup{
            pdfencoding=auto,
            psdextra
        }
        % End of preamble defined in sphinx-jupyterbook-latex %
        

\title{TeachBook Sciences Template}
\date{Feb 09, 2026}
\release{}
\author{Facultad de Ciencias (USAL)}
\newcommand{\sphinxlogo}{\vbox{}}
\renewcommand{\releasename}{}
\makeindex
\begin{document}

\pagestyle{empty}
\sphinxmaketitle
\pagestyle{plain}
\sphinxtableofcontents
\pagestyle{normal}
\phantomsection\label{\detokenize{es/intro::doc}}


\sphinxAtStartPar
Bienvenido a la plantilla \sphinxstylestrong{TeachBook Sciences Template}.

\begin{DUlineblock}{0em}
\item[] \sphinxstylestrong{\Large ¿Qué es esto?}
\end{DUlineblock}

\sphinxAtStartPar
Es una plantilla diseñada para que el profesorado de la \sphinxstylestrong{Facultad de Ciencias de la USAL} pueda crear libros docentes interactivos de forma sencilla.

\begin{DUlineblock}{0em}
\item[] \sphinxstylestrong{\large Contenido}
\end{DUlineblock}

\sphinxAtStartPar
En este libro encontrarás:
\begin{itemize}
\item {} 
\sphinxAtStartPar
{\hyperref[\detokenize{es/01_tutorial/01_que_es_un_teachbook::doc}]{\sphinxcrossref{\DUrole{doc,std,std-doc}{Tutoriales}}}} para aprender a usar la plantilla

\item {} 
\sphinxAtStartPar
{\hyperref[\detokenize{es/02_grados/grado_fisica/intro::doc}]{\sphinxcrossref{\DUrole{doc,std,std-doc}{Ejemplos por Grado}}}} para ver casos reales

\item {} 
\sphinxAtStartPar
Información sobre {\hyperref[\detokenize{es/92_como_citar::doc}]{\sphinxcrossref{\DUrole{doc,std,std-doc}{cómo citar}}}} y {\hyperref[\detokenize{es/91_licencias::doc}]{\sphinxcrossref{\DUrole{doc,std,std-doc}{licencias}}}}

\end{itemize}

\begin{sphinxadmonition}{note}{Note:}
\sphinxAtStartPar
Este proyecto está diseñado para ser usado con \sphinxstylestrong{VS Code} y asistentes de \sphinxstylestrong{IA}.
\end{sphinxadmonition}

\sphinxstepscope


\part{Tutorial}

\sphinxstepscope


\chapter{1. ¿Qué es un TeachBook?}
\label{\detokenize{es/01_tutorial/01_que_es_un_teachbook:que-es-un-teachbook}}\label{\detokenize{es/01_tutorial/01_que_es_un_teachbook::doc}}
\sphinxAtStartPar
Un \sphinxstylestrong{TeachBook} es un libro digital interactivo que combina:
\begin{itemize}
\item {} 
\sphinxAtStartPar
Texto explicativo

\item {} 
\sphinxAtStartPar
Código ejecutable (Python, R, Julia…)

\item {} 
\sphinxAtStartPar
Gráficos interactivos

\item {} 
\sphinxAtStartPar
Ecuaciones matemáticas

\end{itemize}


\section{Ventajas para la docencia}
\label{\detokenize{es/01_tutorial/01_que_es_un_teachbook:ventajas-para-la-docencia}}\begin{enumerate}
\sphinxsetlistlabels{\arabic}{enumi}{enumii}{}{.}%
\item {} 
\sphinxAtStartPar
\sphinxstylestrong{Accesibilidad}: Disponible en web desde cualquier dispositivo.

\item {} 
\sphinxAtStartPar
\sphinxstylestrong{Interactividad}: Los estudiantes pueden modificar el código y ver resultados.

\item {} 
\sphinxAtStartPar
\sphinxstylestrong{Mantenibilidad}: Fácil de actualizar y corregir.

\item {} 
\sphinxAtStartPar
\sphinxstylestrong{Reproducibilidad}: Todo el contenido se genera desde el código fuente.

\end{enumerate}


\section{Tecnología subyacente}
\label{\detokenize{es/01_tutorial/01_que_es_un_teachbook:tecnologia-subyacente}}
\sphinxAtStartPar
Este proyecto utiliza \sphinxhref{https://jupyterbook.org/}{Jupyter Book}, una herramienta estándar en la comunidad científica.

\sphinxstepscope


\chapter{2. Flujo de Trabajo}
\label{\detokenize{es/01_tutorial/02_flujo_trabajo:flujo-de-trabajo}}\label{\detokenize{es/01_tutorial/02_flujo_trabajo::doc}}
\sphinxAtStartPar
Este es el ciclo de vida recomendado para tu TeachBook:
\begin{enumerate}
\sphinxsetlistlabels{\arabic}{enumi}{enumii}{}{.}%
\item {} 
\sphinxAtStartPar
\sphinxstylestrong{Escribir contenido} en archivos Markdown (\sphinxcode{\sphinxupquote{.md}}) o Notebooks (\sphinxcode{\sphinxupquote{.ipynb}}).

\item {} 
\sphinxAtStartPar
\sphinxstylestrong{Previsualizar} los cambios localmente (opcional, pero recomendado).

\item {} 
\sphinxAtStartPar
\sphinxstylestrong{Guardar versiones} (commit) con Git.

\item {} 
\sphinxAtStartPar
\sphinxstylestrong{Publicar} (push) a GitHub para que la web se actualice automáticamente.

\end{enumerate}


\section{Estructura de archivos}
\label{\detokenize{es/01_tutorial/02_flujo_trabajo:estructura-de-archivos}}
\sphinxAtStartPar
El contenido se organiza en la carpeta \sphinxcode{\sphinxupquote{book/es/}}.
\begin{itemize}
\item {} 
\sphinxAtStartPar
\sphinxcode{\sphinxupquote{intro.md}}: Página principal.

\item {} 
\sphinxAtStartPar
\sphinxcode{\sphinxupquote{\_toc.yml}}: Índice del libro.

\item {} 
\sphinxAtStartPar
\sphinxcode{\sphinxupquote{\_config.yml}}: Configuración general.

\end{itemize}

\sphinxstepscope


\chapter{3. Edición con IA}
\label{\detokenize{es/01_tutorial/03_edicion_con_ia:edicion-con-ia}}\label{\detokenize{es/01_tutorial/03_edicion_con_ia::doc}}
\sphinxAtStartPar
Este template está diseñado para usarse con asistentes de IA como \sphinxstylestrong{GitHub Copilot} o \sphinxstylestrong{Antigravity}.


\section{Cómo pedir ayuda a la IA}
\label{\detokenize{es/01_tutorial/03_edicion_con_ia:como-pedir-ayuda-a-la-ia}}
\sphinxAtStartPar
Puedes pedirle cosas como:
\begin{itemize}
\item {} 
\sphinxAtStartPar
“Explícame este código de Python”

\item {} 
\sphinxAtStartPar
“Genera un gráfico de una función seno”

\item {} 
\sphinxAtStartPar
“Revisa la ortografía de este párrafo”

\item {} 
\sphinxAtStartPar
“Añade una ecuación matemática en formato LaTeX”

\end{itemize}


\section{Ejemplo de Prompt}
\label{\detokenize{es/01_tutorial/03_edicion_con_ia:ejemplo-de-prompt}}\begin{quote}

\sphinxAtStartPar
“Crea una celda de código que grafique la distribución normal estándar usando matplotlib.”
\end{quote}

\sphinxAtStartPar
La IA te dará el código y tú solo tienes que ejecutarlo o incluirlo en tu libro.

\sphinxstepscope


\chapter{4. Publicación Web}
\label{\detokenize{es/01_tutorial/04_publicacion:publicacion-web}}\label{\detokenize{es/01_tutorial/04_publicacion::doc}}
\sphinxAtStartPar
La publicación se realiza de forma automática a través de \sphinxstylestrong{GitHub Pages}.


\section{Pasos para publicar}
\label{\detokenize{es/01_tutorial/04_publicacion:pasos-para-publicar}}\begin{enumerate}
\sphinxsetlistlabels{\arabic}{enumi}{enumii}{}{.}%
\item {} 
\sphinxAtStartPar
Asegúrate de que tu repositorio es público (o tienes acceso a GitHub Pages en privado).

\item {} 
\sphinxAtStartPar
Haz cambios en tu contenido.

\item {} 
\sphinxAtStartPar
Guarda los cambios un \sphinxstylestrong{Commit} y haz \sphinxstylestrong{Push} a la rama principal (\sphinxcode{\sphinxupquote{main}}).

\item {} 
\sphinxAtStartPar
Una “Acción de GitHub” se ejecutará automáticamente y construirá tu libro.

\item {} 
\sphinxAtStartPar
En unos minutos, verás tu libro actualizado en la URL de tu repositorio (configurada en Settings > Pages).

\end{enumerate}

\sphinxstepscope


\part{Ejemplos por Grado}

\sphinxstepscope


\chapter{Grado en Física}
\label{\detokenize{es/02_grados/grado_fisica/intro:grado-en-fisica}}\label{\detokenize{es/02_grados/grado_fisica/intro::doc}}
\sphinxAtStartPar
En esta sección encontrarás ejemplos adaptados a asignaturas del Grado en Física.
El objetivo es mostrar cómo se pueden integrar:
\begin{itemize}
\item {} 
\sphinxAtStartPar
Fórmulas matemáticas complejas

\item {} 
\sphinxAtStartPar
Gráficos de simulaciones físicas

\item {} 
\sphinxAtStartPar
Código de análisis de datos experimentales

\end{itemize}

\sphinxstepscope


\section{Ejemplo de Física: Oscilador Armónico}
\label{\detokenize{es/02_grados/grado_fisica/ejemplo_fisica:ejemplo-de-fisica-oscilador-armonico}}\label{\detokenize{es/02_grados/grado_fisica/ejemplo_fisica::doc}}
\sphinxAtStartPar
Este notebook simula un oscilador armónico simple.

\begin{sphinxuseclass}{cell}\begin{sphinxVerbatimInput}

\begin{sphinxuseclass}{cell_input}
\begin{sphinxVerbatim}[commandchars=\\\{\}]
\PYG{k+kn}{import} \PYG{n+nn}{numpy} \PYG{k}{as} \PYG{n+nn}{np}
\PYG{k+kn}{import} \PYG{n+nn}{matplotlib}\PYG{n+nn}{.}\PYG{n+nn}{pyplot} \PYG{k}{as} \PYG{n+nn}{plt}

\PYG{c+c1}{\PYGZsh{} Parámetros}
\PYG{n}{k} \PYG{o}{=} \PYG{l+m+mf}{1.0}  \PYG{c+c1}{\PYGZsh{} Constante del resorte}
\PYG{n}{m} \PYG{o}{=} \PYG{l+m+mf}{1.0}  \PYG{c+c1}{\PYGZsh{} Masa}
\PYG{n}{omega} \PYG{o}{=} \PYG{n}{np}\PYG{o}{.}\PYG{n}{sqrt}\PYG{p}{(}\PYG{n}{k} \PYG{o}{/} \PYG{n}{m}\PYG{p}{)}
\PYG{n}{t} \PYG{o}{=} \PYG{n}{np}\PYG{o}{.}\PYG{n}{linspace}\PYG{p}{(}\PYG{l+m+mi}{0}\PYG{p}{,} \PYG{l+m+mi}{20}\PYG{p}{,} \PYG{l+m+mi}{100}\PYG{p}{)}

\PYG{c+c1}{\PYGZsh{} Posición}
\PYG{n}{x} \PYG{o}{=} \PYG{n}{np}\PYG{o}{.}\PYG{n}{cos}\PYG{p}{(}\PYG{n}{omega} \PYG{o}{*} \PYG{n}{t}\PYG{p}{)}

\PYG{c+c1}{\PYGZsh{} Gráfico}
\PYG{n}{plt}\PYG{o}{.}\PYG{n}{figure}\PYG{p}{(}\PYG{n}{figsize}\PYG{o}{=}\PYG{p}{(}\PYG{l+m+mi}{8}\PYG{p}{,} \PYG{l+m+mi}{4}\PYG{p}{)}\PYG{p}{)}
\PYG{n}{plt}\PYG{o}{.}\PYG{n}{plot}\PYG{p}{(}\PYG{n}{t}\PYG{p}{,} \PYG{n}{x}\PYG{p}{)}
\PYG{n}{plt}\PYG{o}{.}\PYG{n}{title}\PYG{p}{(}\PYG{l+s+s1}{\PYGZsq{}}\PYG{l+s+s1}{Oscilador Armónico Simple}\PYG{l+s+s1}{\PYGZsq{}}\PYG{p}{)}
\PYG{n}{plt}\PYG{o}{.}\PYG{n}{xlabel}\PYG{p}{(}\PYG{l+s+s1}{\PYGZsq{}}\PYG{l+s+s1}{Tiempo (s)}\PYG{l+s+s1}{\PYGZsq{}}\PYG{p}{)}
\PYG{n}{plt}\PYG{o}{.}\PYG{n}{ylabel}\PYG{p}{(}\PYG{l+s+s1}{\PYGZsq{}}\PYG{l+s+s1}{Posición (m)}\PYG{l+s+s1}{\PYGZsq{}}\PYG{p}{)}
\PYG{n}{plt}\PYG{o}{.}\PYG{n}{grid}\PYG{p}{(}\PYG{k+kc}{True}\PYG{p}{)}
\PYG{n}{plt}\PYG{o}{.}\PYG{n}{show}\PYG{p}{(}\PYG{p}{)}
\end{sphinxVerbatim}

\end{sphinxuseclass}\end{sphinxVerbatimInput}

\end{sphinxuseclass}
\sphinxstepscope


\chapter{Grado en Matemáticas}
\label{\detokenize{es/02_grados/grado_matematicas/intro:grado-en-matematicas}}\label{\detokenize{es/02_grados/grado_matematicas/intro::doc}}
\sphinxAtStartPar
Ejemplos para asignaturas de matemáticas.
Aquí se prioriza:
\begin{itemize}
\item {} 
\sphinxAtStartPar
Rigor en la notación matemática (LaTeX)

\item {} 
\sphinxAtStartPar
Demostraciones visuales

\item {} 
\sphinxAtStartPar
Algoritmos simbólicos (SymPy)

\end{itemize}

\sphinxstepscope


\section{Ejemplo de Matemáticas: Cálculo Simbólico}
\label{\detokenize{es/02_grados/grado_matematicas/ejemplo_matematicas:ejemplo-de-matematicas-calculo-simbolico}}\label{\detokenize{es/02_grados/grado_matematicas/ejemplo_matematicas::doc}}
\sphinxAtStartPar
Vamos a usar SymPy para derivar e integrar funciones.

\begin{sphinxuseclass}{cell}\begin{sphinxVerbatimInput}

\begin{sphinxuseclass}{cell_input}
\begin{sphinxVerbatim}[commandchars=\\\{\}]
\PYG{k+kn}{from} \PYG{n+nn}{sympy} \PYG{k+kn}{import} \PYG{n}{symbols}\PYG{p}{,} \PYG{n}{diff}\PYG{p}{,} \PYG{n}{integrate}\PYG{p}{,} \PYG{n}{sin}\PYG{p}{,} \PYG{n}{exp}

\PYG{n}{x} \PYG{o}{=} \PYG{n}{symbols}\PYG{p}{(}\PYG{l+s+s1}{\PYGZsq{}}\PYG{l+s+s1}{x}\PYG{l+s+s1}{\PYGZsq{}}\PYG{p}{)}
\PYG{n}{f} \PYG{o}{=} \PYG{n}{exp}\PYG{p}{(}\PYG{o}{\PYGZhy{}}\PYG{n}{x}\PYG{p}{)} \PYG{o}{*} \PYG{n}{sin}\PYG{p}{(}\PYG{n}{x}\PYG{p}{)}

\PYG{c+c1}{\PYGZsh{} Derivada}
\PYG{n}{df} \PYG{o}{=} \PYG{n}{diff}\PYG{p}{(}\PYG{n}{f}\PYG{p}{,} \PYG{n}{x}\PYG{p}{)}
\PYG{n+nb}{print}\PYG{p}{(}\PYG{l+s+sa}{f}\PYG{l+s+s2}{\PYGZdq{}}\PYG{l+s+s2}{Derivada: }\PYG{l+s+si}{\PYGZob{}}\PYG{n}{df}\PYG{l+s+si}{\PYGZcb{}}\PYG{l+s+s2}{\PYGZdq{}}\PYG{p}{)}

\PYG{c+c1}{\PYGZsh{} Integral indefinida}
\PYG{n}{int\PYGZus{}f} \PYG{o}{=} \PYG{n}{integrate}\PYG{p}{(}\PYG{n}{f}\PYG{p}{,} \PYG{n}{x}\PYG{p}{)}
\PYG{n+nb}{print}\PYG{p}{(}\PYG{l+s+sa}{f}\PYG{l+s+s2}{\PYGZdq{}}\PYG{l+s+s2}{Integral: }\PYG{l+s+si}{\PYGZob{}}\PYG{n}{int\PYGZus{}f}\PYG{l+s+si}{\PYGZcb{}}\PYG{l+s+s2}{\PYGZdq{}}\PYG{p}{)}
\end{sphinxVerbatim}

\end{sphinxuseclass}\end{sphinxVerbatimInput}

\end{sphinxuseclass}
\sphinxstepscope


\chapter{Grado en Estadística}
\label{\detokenize{es/02_grados/grado_estadistica/intro:grado-en-estadistica}}\label{\detokenize{es/02_grados/grado_estadistica/intro::doc}}
\sphinxAtStartPar
Ejemplos orientados al análisis de datos y probabilidad.
Destacamos:
\begin{itemize}
\item {} 
\sphinxAtStartPar
Manejo de DataFrames (Pandas)

\item {} 
\sphinxAtStartPar
Visualización estadística (Seaborn)

\item {} 
\sphinxAtStartPar
Modelos probabilísticos

\end{itemize}

\sphinxstepscope


\section{Ejemplo de Estadística: Generación de Datos}
\label{\detokenize{es/02_grados/grado_estadistica/ejemplo_estadistica:ejemplo-de-estadistica-generacion-de-datos}}\label{\detokenize{es/02_grados/grado_estadistica/ejemplo_estadistica::doc}}
\sphinxAtStartPar
Generamos una distribución normal y visualizamos su histograma.

\begin{sphinxuseclass}{cell}\begin{sphinxVerbatimInput}

\begin{sphinxuseclass}{cell_input}
\begin{sphinxVerbatim}[commandchars=\\\{\}]
\PYG{k+kn}{import} \PYG{n+nn}{numpy} \PYG{k}{as} \PYG{n+nn}{np}
\PYG{k+kn}{import} \PYG{n+nn}{matplotlib}\PYG{n+nn}{.}\PYG{n+nn}{pyplot} \PYG{k}{as} \PYG{n+nn}{plt}

\PYG{n}{data} \PYG{o}{=} \PYG{n}{np}\PYG{o}{.}\PYG{n}{random}\PYG{o}{.}\PYG{n}{normal}\PYG{p}{(}\PYG{l+m+mi}{0}\PYG{p}{,} \PYG{l+m+mi}{1}\PYG{p}{,} \PYG{l+m+mi}{1000}\PYG{p}{)}

\PYG{n}{plt}\PYG{o}{.}\PYG{n}{hist}\PYG{p}{(}\PYG{n}{data}\PYG{p}{,} \PYG{n}{bins}\PYG{o}{=}\PYG{l+m+mi}{30}\PYG{p}{,} \PYG{n}{alpha}\PYG{o}{=}\PYG{l+m+mf}{0.7}\PYG{p}{,} \PYG{n}{color}\PYG{o}{=}\PYG{l+s+s1}{\PYGZsq{}}\PYG{l+s+s1}{green}\PYG{l+s+s1}{\PYGZsq{}}\PYG{p}{,} \PYG{n}{density}\PYG{o}{=}\PYG{k+kc}{True}\PYG{p}{)}
\PYG{n}{plt}\PYG{o}{.}\PYG{n}{title}\PYG{p}{(}\PYG{l+s+s1}{\PYGZsq{}}\PYG{l+s+s1}{Histograma de Distribución Normal}\PYG{l+s+s1}{\PYGZsq{}}\PYG{p}{)}
\PYG{n}{plt}\PYG{o}{.}\PYG{n}{show}\PYG{p}{(}\PYG{p}{)}
\end{sphinxVerbatim}

\end{sphinxuseclass}\end{sphinxVerbatimInput}

\end{sphinxuseclass}
\sphinxstepscope


\part{Información}

\sphinxstepscope


\chapter{Acerca de}
\label{\detokenize{es/90_acerca_de:acerca-de}}\label{\detokenize{es/90_acerca_de::doc}}

\section{Autores}
\label{\detokenize{es/90_acerca_de:autores}}
\sphinxAtStartPar
\sphinxstylestrong{Facultad de Ciencias}
Universidad de Salamanca (USAL)


\section{Contexto}
\label{\detokenize{es/90_acerca_de:contexto}}
\sphinxAtStartPar
Este libro forma parte del proyecto de innovación docente para la integración de herramientas digitales en la enseñanza de las ciencias.


\section{Año}
\label{\detokenize{es/90_acerca_de:ano}}
\sphinxAtStartPar
2025

\sphinxstepscope


\chapter{Licencias}
\label{\detokenize{es/91_licencias:licencias}}\label{\detokenize{es/91_licencias::doc}}

\section{Contenidos}
\label{\detokenize{es/91_licencias:contenidos}}
\sphinxAtStartPar
Todo el contenido de texto e imágenes (salvo que se indique lo contrario) se distribuye bajo licencia \sphinxstylestrong{Creative Commons Atribución 4.0 Internacional (CC BY 4.0)}.


\section{Código}
\label{\detokenize{es/91_licencias:codigo}}
\sphinxAtStartPar
Los ejemplos de código fuente se distribuyen bajo licencia \sphinxstylestrong{MIT}.


\section{Atribuciones}
\label{\detokenize{es/91_licencias:atribuciones}}
\sphinxAtStartPar
Este proyecto utiliza:
\begin{itemize}
\item {} 
\sphinxAtStartPar
\sphinxhref{https://jupyterbook.org/}{Jupyter Book}

\item {} 
\sphinxAtStartPar
\sphinxhref{https://the-turing-way.netlify.app/}{The Turing Way} (como inspiración)

\end{itemize}

\sphinxstepscope


\chapter{Cómo citar}
\label{\detokenize{es/92_como_citar:como-citar}}\label{\detokenize{es/92_como_citar::doc}}
\sphinxAtStartPar
Si utilizas este material, por favor cítalo de la siguiente manera:


\section{Cita textual}
\label{\detokenize{es/92_como_citar:cita-textual}}\begin{quote}

\sphinxAtStartPar
Facultad de Ciencias (2025). \sphinxstyleemphasis{TeachBook Sciences Template}. Universidad de Salamanca. Disponible en: {[}URL del repositorio{]}
\end{quote}


\section{BibTeX}
\label{\detokenize{es/92_como_citar:bibtex}}
\begin{sphinxVerbatim}[commandchars=\\\{\}]
\PYG{n+nc}{@book}\PYG{p}{\PYGZob{}}\PYG{n+nl}{teachbook\PYGZus{}sciences\PYGZus{}2025}\PYG{p}{,}
\PYG{+w}{  }\PYG{n+na}{author}\PYG{+w}{ }\PYG{p}{=}\PYG{+w}{ }\PYG{l+s}{\PYGZob{}}\PYG{l+s}{Facultad de Ciencias}\PYG{l+s}{\PYGZcb{}}\PYG{p}{,}
\PYG{+w}{  }\PYG{n+na}{title}\PYG{+w}{ }\PYG{p}{=}\PYG{+w}{ }\PYG{l+s}{\PYGZob{}}\PYG{l+s}{TeachBook Sciences Template}\PYG{l+s}{\PYGZcb{}}\PYG{p}{,}
\PYG{+w}{  }\PYG{n+na}{year}\PYG{+w}{ }\PYG{p}{=}\PYG{+w}{ }\PYG{l+s}{\PYGZob{}}\PYG{l+s}{2025}\PYG{l+s}{\PYGZcb{}}\PYG{p}{,}
\PYG{+w}{  }\PYG{n+na}{publisher}\PYG{+w}{ }\PYG{p}{=}\PYG{+w}{ }\PYG{l+s}{\PYGZob{}}\PYG{l+s}{Universidad de Salamanca}\PYG{l+s}{\PYGZcb{}}\PYG{p}{,}
\PYG{+w}{  }\PYG{n+na}{url}\PYG{+w}{ }\PYG{p}{=}\PYG{+w}{ }\PYG{l+s}{\PYGZob{}}\PYG{l+s}{https://github.com/usuario/repo}\PYG{l+s}{\PYGZcb{}}
\PYG{p}{\PYGZcb{}}
\end{sphinxVerbatim}


\section{DOI}
\label{\detokenize{es/92_como_citar:doi}}
\sphinxAtStartPar
(Pendiente de asignación en Zenodo)







\renewcommand{\indexname}{Index}
\printindex
\end{document}